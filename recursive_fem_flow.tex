\documentclass{article}
\usepackage{amsmath} % For using \text and other math functions
\usepackage{amsfonts} % For using \mathbb and other math fonts
\usepackage{graphicx} % For including graphics

\title{Recursive Finite Element Method (RFEM) Flow}
\author{}
\date{}

\begin{document}

\maketitle

\section*{Recursive Finite Element Method (RFEM) Flow}

The Recursive Finite Element Method (RFEM) provides a structured approach for solving differential equations in engineering analysis, particularly focusing on the recursive assembly of matrices.

\subsection*{Key Steps}

\begin{enumerate}
    \item \textbf{Finite Element Discretization}:
    \begin{itemize}
        \item Convert the governing differential equation into a weak form suitable for discretization.
        \item Represent the solution using \textbf{shape functions} \( N \) over each element.
    \end{itemize}

    \item \textbf{Element Matrices}:
    \begin{itemize}
        \item Define element-wise \textbf{stiffness matrix} \( K_e \), \textbf{mass matrix} \( M_e \), and \textbf{load vector} \( F_e \).
        \item Assemble the \textbf{global stiffness matrix} \( K \) recursively:
        \[
        K = \sum_{e=1}^{n} K_e
        \]
        \item Recursively add each element's contribution:
        \[
        K^{(r)} = K^{(r-1)} + K_e
        \]
    \end{itemize}

    \item \textbf{Boundary Condition Application}:
    \begin{itemize}
        \item Adjust the global matrix to enforce \textbf{boundary conditions}:
        \[
        K_{ii} = 1, \quad K_{ij} = 0 \; (i \neq j), \quad F_i = u_i
        \]
    \end{itemize}

    \item \textbf{System Solution}:
    \begin{itemize}
        \item Solve the linear system:
        \[
        K u = F
        \]
        \item Use methods like \textbf{conjugate gradient} for efficient recursive solution, especially for large, sparse matrices.
    \end{itemize}

    \item \textbf{Adaptive Refinement}:
    \begin{itemize}
        \item Use \textbf{error estimation} to refine elements:
        \[
        e_e = \left\lVert u_e - u_{\text{exact}} \right\rVert > \text{threshold}
        \]
        \item Recursively refine elements with high error to improve accuracy.
    \end{itemize}
\end{enumerate}

\subsection*{Summary}
RFEM involves recursive matrix assembly, application of boundary conditions, solving the linear system, and adaptive refinement based on error estimates. This approach is well-suited for large-scale engineering problems where efficient handling of matrices and error control is crucial.

\end{document}
